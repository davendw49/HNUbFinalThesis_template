% !Mode:: "TeX:UTF-8"
%  Authors: 杜家宜   Jiayi Du: Max_dujiayi@gmail.com     湖南大学2010级计算机科学与技术专业博士生

%%%%%%%%%% Fonts Definition and Basics %%%%%%%%%%%%%%%%%
\newcommand{\song}{\CJKfamily{zhsong}}    % 宋体
\newcommand{\fs}{\CJKfamily{zhfs}}        % 仿宋体
\newcommand{\kai}{\CJKfamily{zhkai}}      % 楷体
\newcommand{\hei}{\CJKfamily{zhhei}}      % 黑体
\newcommand{\li}{\CJKfamily{zhli}}        % 隶书
\newcommand{\biaoti}{\fontsize{45pt}{45pt}\selectfont}       % 一号, 1.倍行距
\newcommand{\yihao}{\fontsize{26pt}{26pt}\selectfont}       % 一号, 1.倍行距
\newcommand{\xiaoyi}{\fontsize{24pt}{24pt}\selectfont}      % 小一, 1.倍行距
\newcommand{\erhao}{\fontsize{22pt}{22pt}\selectfont}       % 二号, 1.倍行距
\newcommand{\xiaoer}{\fontsize{18pt}{18pt}\selectfont}      % 小二, 单倍行距
\newcommand{\sanhao}{\fontsize{16pt}{16pt}\selectfont}      % 三号, 1.倍行距
\newcommand{\xiaosan}{\fontsize{15pt}{15pt}\selectfont}     % 小三, 1.倍行距
\newcommand{\sihao}{\fontsize{14pt}{14pt}\selectfont}       % 四号, 1.0倍行距
\newcommand{\xiaosi}{\fontsize{12.5pt}{12.5pt}\selectfont}  % 小四, 1.倍行距
\newcommand{\wuhao}{\fontsize{10.5pt}{10.5pt}\selectfont}   % 五号, 单倍行距
\newcommand{\xiaowu}{\fontsize{9pt}{9pt}\selectfont}        % 小五, 单倍行距
\setlength{\headheight}{20pt}
%\CJKcaption{gb_452}
\CJKtilde  % 重新定义了波浪符~的意义
\newcommand\prechaptername{第}
\newcommand\postchaptername{章}

% 调整罗列环境的布局 % Modified by Li Jianmin
\setitemize{leftmargin=0em,itemindent=3em,itemsep=0em,partopsep=0em,parsep=0em,topsep=-0em}
\setenumerate{leftmargin=0em,itemindent=3em,itemsep=0em,partopsep=0em,parsep=0em,topsep=0em}


%避免宏包 hyperref 和 arydshln 不兼容带来的目录链接失效的问题。
\def\temp{\relax}
\let\temp\addcontentsline
\gdef\addcontentsline{\phantomsection\temp}

% 自定义项目列表标签及格式 \begin{publist} 列表项 \end{publist}
\newcounter{pubctr} % 自定义新计数器
\newenvironment{publist}{%%%%% 定义新环境
\begin{list}{[\arabic{pubctr}]} %% 标签格式
    {
     \usecounter{pubctr}
     \setlength{\leftmargin}{2em}     % 左边界 \leftmargin =\itemindent + \labelwidth + \labelsep
     \setlength{\itemindent}{0em}     % 标号缩进量
     \setlength{\labelsep}{1em}       % 标号和列表项之间的距离,默认0.5em
     \setlength{\rightmargin}{0em}    % 右边界
     \setlength{\topsep}{0ex}         % 列表到上下文的垂直距离
     \setlength{\parsep}{0ex}         % 段落间距
     \setlength{\itemsep}{0ex}        % 标签间距
     \setlength{\listparindent}{0pt} % 段落缩进量
    }}
{\end{list}}%%%%%


\makeatletter
\renewcommand\normalsize{
  \@setfontsize\normalsize{12.5pt}{12.5pt} % 小四对应12pt
  \setlength\abovedisplayskip{4pt}
  \setlength\abovedisplayshortskip{4pt}
  \setlength\belowdisplayskip{\abovedisplayskip}
  \setlength\belowdisplayshortskip{\abovedisplayshortskip}
    \let\@listi\@listI}
\def\defaultfont{\renewcommand{\baselinestretch}{1.65}\normalsize\selectfont}


% 设置行距和段落间垂直距离

\setlength{\baselineskip}{20pt}
\renewcommand{\CJKglue}{\hskip 0.5pt plus \baselineskip} %加大字间距,使每行35个字


\makeatother

%%%%%%%%%%%%% Contents %%%%%%%%%%%%%%%%%
\renewcommand{\contentsname}{目\quad录}
\setcounter{tocdepth}{2}
\titlecontents{chapter}[0em]{\xiaosi\hei}%
             {\prechaptername~~\thecontentslabel~~\postchaptername~~~}{} %
             {\titlerule*[5pt]{$\cdot$}\xiaosi\contentspage}
\titlecontents{section}[2.5em]{\xiaosi\song} %
            {\thecontentslabel\quad}{} %
            {\hspace{.25em}\titlerule*[5pt]{$\cdot$}\xiaosi\contentspage}
\titlecontents{subsection}[3.25em]{\xiaosi\song} %
            {\thecontentslabel\quad}{} %
            {\hspace{.25em}\titlerule*[5pt]{$\cdot$}\xiaosi\contentspage}
\renewcommand{\cftdotsep}{1.1}
\renewcommand{\listfigurename}{插图索引}
\setcounter{lofdepth}{1}
%\titlefigures{chapter}[1em]{\xiaosi\hei}%
%             {\prechaptername~~\thecontentslabel~~\postchaptername~~~}{} %
%            {\titlerule*[5pt]{$\cdot$}\xiaosi\contentspage}
\renewcommand{\listtablename}{附表索引}


%%删除表格和插图因章不同中的空行%%%
\makeatletter
\def\@chapter[#1]#2{\ifnum \c@secnumdepth >\m@ne
                       \if@mainmatter
                         \refstepcounter{chapter}%
                         \typeout{\@chapapp\space\thechapter.}%
                         \addcontentsline{toc}{chapter}%
                                   {\protect\numberline{\thechapter}#1}%
                       \else
                         \addcontentsline{toc}{chapter}{#1}%
                       \fi
                    \else
                      \addcontentsline{toc}{chapter}{#1}%
                    \fi
                    \chaptermark{#1}%
                    \if@twocolumn
                      \@topnewpage[\@makechapterhead{#2}]%
                    \else
                      \@makechapterhead{#2}%
                      \@afterheading
                    \fi}
\makeatother


%%%%%%%%%% Chapter and Section %%%%%%%%%%%%%%%%%
\setcounter{secnumdepth}{4}
\setlength{\parindent}{2em}
\renewcommand{\chaptername}{\prechaptername\arabic{chapter}\postchaptername}
\titleformat{\chapter}{\centering\xiaoer\hei}{\chaptername}{1em}{}
\titlespacing{\chapter}{0pt}{0pt}{18pt}
\titleformat{\section}{\xiaosan\hei}{\thesection}{1em}{}
\titlespacing{\section}{0pt}{12pt}{12pt}
\titleformat{\subsection}{\sihao\hei}{\thesubsection}{0.5em}{}
\titlespacing{\subsection}{0pt}{6pt}{6pt}
\titleformat{\subsubsection}{\xiaosi\hei}{\thesubsubsection}{0.5em}{}
\titlespacing{\subsubsection}{0pt}{6pt}{6pt}

%%%%%%%%%% Table, Figure and Equation %%%%%%%%%%%%%%%%%
\renewcommand{\tablename}{表} % 插表题头
\renewcommand{\figurename}{图} % 插图题头
\renewcommand{\thefigure}{\arabic{chapter}.\arabic{figure}} % 使图编号为 7.1 的格式 %\protect{~}
\renewcommand{\thetable}{\arabic{chapter}.\arabic{table}}%使表编号为 7.1 的格式
\renewcommand{\theequation}{\arabic{chapter}.\arabic{equation}}%使公式编号为 7.1 的格式
\renewcommand{\thesubfigure}{\alph{subfigure})}%使子图编号为a)的格式
\renewcommand{\thesubtable}{(\alph{subtable})} %使子表编号为a)的格式
\makeatletter
\renewcommand{\p@subfigure}{\thefigure~} %使子图引用为 7.1 a) 的格式,母图编号和子图编号之间用~ 加一个空格
\makeatother


%% 定制浮动图形和表格标题样式
\makeatletter
\long\def\@makecaption#1#2{%
   \vskip\abovecaptionskip
   \sbox\@tempboxa{\centering\wuhao\hei{#1~~#2} }%
   \ifdim \wd\@tempboxa >\hsize
     \centering\wuhao\hei{#1~~#2} \par
   \else
     \global \@minipagefalse
     \hb@xt@\hsize{\hfil\box\@tempboxa\hfil}%
   \fi
   \vskip\belowcaptionskip}
\makeatother
\captiondelim{~~~~} %用来控制longtable表头分隔符

%%%%%%%%%% Theorem Environment %%%%%%%%%%%%%%%%%
\theoremstyle{plain}
\theorembodyfont{\xiaosi\song}%\rmfamily}
\theoremheaderfont{\xiaosi\hei}%\rmfamily}
\setlength{\theorempreskipamount}{0em} %调整定理环境与上文的距离
\setlength{\theorempostskipamount}{0em} %调整定理环境与下文的距离
\newtheorem{theorem}{定理~}[chapter]
\newtheorem{lemma}{引理~}[chapter]
\newtheorem{axiom}{公理~}[chapter]
\newtheorem{proposition}{命题~}[chapter]
\newtheorem{corollary}{推论~}[chapter]
\newtheorem{definition}{\hskip 2em 定义~}[chapter]
\newtheorem{conjecture}{猜想~}[chapter]
\newtheorem{example}{例~}[chapter]
\newtheorem{remark}{注~}[chapter]
\floatname{algorithm}{算法}% 将英文的algorithm改为算法
\renewcommand{\algorithmicrequire}{\textbf{Input:}}
\renewcommand{\algorithmicensure}{\textbf{Output:}}
\newcommand{\tabincell}[2]{\begin{tabular}{@{}#1@{}}#2\end{tabular}}%表格合并
\newenvironment{proof}{\noindent{\hei 证明:}}{\hfill $ \square $ \vskip 4mm}
\theoremsymbol{$\square$}

%%%%%%%%%% Page: number, header and footer  页码%%%%%%%%%%%%%%%%%

%\frontmatter 或 \pagenumbering{roman}
%\mainmatter 或 \pagenumbering{arabic}
\makeatletter
\renewcommand\frontmatter{\clearpage
  \@mainmatterfalse
  \pagenumbering{Roman}} % 正文前罗马字体编号
\makeatother


%%%%%%%%%% References %%%%%%%%%%%%%%%%%
\renewcommand{\bibname}{参考文献}
% 重定义参考文献样式,来自thu
\makeatletter
\renewenvironment{thebibliography}[1]{%
   \chapter*{\bibname}%
   \xiaosi
   \list{\@biblabel{\@arabic\c@enumiv}}%
        {\renewcommand{\makelabel}[1]{##1\hfill}
         \setlength{\baselineskip}{21pt}
         \settowidth\labelwidth{0.5cm}
         \setlength{\labelsep}{0pt}
         \setlength{\itemindent}{0pt}
         \setlength{\leftmargin}{\labelwidth+\labelsep}
         \addtolength{\itemsep}{-0.7em}
         \usecounter{enumiv}%
         \let\p@enumiv\@empty
         \renewcommand\theenumiv{\@arabic\c@enumiv}}%
    \sloppy\frenchspacing
    \clubpenalty4000%
    \@clubpenalty \clubpenalty
    \widowpenalty4000%
    \interlinepenalty4000%
    \sfcode`\.\@m}
   {\def\@noitemerr
     {\@latex@warning{Empty `thebibliography' environment}}%
    \endlist\frenchspacing}
\makeatother

\addtolength{\bibsep}{5pt} % 增加参考文献间的垂直间距
\setlength{\bibhang}{2em} %每个条目自第二行起缩进的距离

% 参考文献引用作为上标出现
\newcommand{\mycite}[1]{\scalebox{1.3}[1.3]{\raisebox{-0.65ex}{\cite{#1}}}}
%\newcommand{\citenormal}[1]{\cite{#1}}
%\makeatletter
%   \def\@cite#1#2{\textsuperscript{[{#1\if@tempswa , #2\fi}]}}
%\makeatother

%% 引用格式
\bibpunct{[}{]}{,}{s}{}{,}

%%%%%%%%%% Cover %%%%%%%%%%%%%%%%%
% 封面、摘要、版权、致谢格式定义
\makeatletter

\def\title#1{\def\@title{#1}}\def\@title{}
\def\titlelineOne#1{\def\@titlelineOne{#1}}\def\@titlelineOne{}
\def\titlelineTwo#1{\def\@titlelineTwo{#1}}\def\@titlelineTwo{}
\def\affil#1{\def\@affil{#1}}\def\@affil{}
\def\major#1{\def\@major{#1}}\def\@major{}
\def\author#1{\def\@author{#1}}\def\@author{}
\def\teacher#1{\def\@teacher{#1}}\def\@teacher{}
\def\dean#1{\def\@dean{#1}}\def\@dean{}
\def\stdnumber#1{\def\@stdnumber{#1}}\def\@stdnumber{}
\def\date#1{\def\@date{#1}}\def\@date{}
\def\untitle#1{\def\@untitle{#1}}\def\@untitle{}
\def\declaretitle#1{\def\@declaretitle{#1}}\def\@declaretitle{}
\def\declarecontent#1{\def\@declarecontent{#1}}\def\@declarecontent{}
\def\authorizationtitle#1{\def\@authorizationtitle{#1}}\def\@authorizationtitle{}
\def\authorizationcontent#1{\def\@authorizationcontent{#1}}\def\@authorizationconent{}
\def\authorizationadd#1{\def\@authorizationadd{#1}}\def\@authorizationadd{}
\def\authorsigncap#1{\def\@authorsigncap{#1}}\def\@authorsigncap{}
\def\supervisorsigncap#1{\def\@supervisorsigncap{#1}}\def\@supervisorsigncap{}
\def\signdatecap#1{\def\@signdatecap{#1}}\def\@signdatecap{}
\long\def\cabstract#1{\long\def\@cabstract{#1}}\long\def\@cabstract{}
\long\def\eabstract#1{\long\def\@eabstract{#1}}\long\def\@eabstract{}
\def\ckeywords#1{\def\@ckeywords{#1}}\def\@ckeywords{}
\def\ekeywords#1{\def\@ekeywords{#1}}\def\@ekeywords{}
\def\heading#1{\def\@heading{#1}}\def\@heading{}
\def\headtitle#1{\def\@headtitle{#1}}\def\@headtitle{}


\newlength{\@title@width}
\def\@put@covertitle#1{\makebox[\@title@width][s]{#1}}
% 定义封面
\def\makecover{
	%\cleardoublepage%
	\phantomsection
	\pdfbookmark[-1]{\@title}{title}
	\newgeometry{left=2cm,right=2cm,top=2.0cm,bottom=2cm}
	\begin{titlepage}
		\begin{center}
			
			\setlength{\@title@width}{3.5cm}
			
			
			\begin{figure}[h]
				\centering
				\includegraphics[width=0.5\textwidth]{figures/hnu}
			\end{figure}
			
			
			\begin{overpic}[scale=0.30]{figures/Hnulogo}
				\put(8.9,75){\bfseries\yihao HUNAN UNIVERSITY}
				\put(-14.9,45){\song\biaoti \@heading}
			\end{overpic}
			
			\vspace*{1cm}
			
			\vspace{\baselineskip}
			\setlength{\@title@width}{6.5cm}
			{
				
				\begin{spacing}{2.1}
					\xiaoer\hei{论文题目:} \xiaoer \hei\underline{\makebox[\@title@width][c]{\@titlelineOne}} \\
					\xiaoer\hei{\qquad\qquad\quad} \xiaoer \hei\underline{\makebox[\@title@width][c]{\@titlelineTwo}} \\
					\xiaosi\hei{学~生~姓~名:} \xiaosi\song\underline{\makebox[\@title@width][c]{\@author}} \\
					\xiaosi\hei{学~生~学~号:} \xiaosi\song\underline{\makebox[\@title@width][c]{\@stdnumber}} \\
					\xiaosi\hei{专~业~班~级:} \xiaosi\song\underline{\makebox[\@title@width][c]{\@major}} \\
					\xiaosi\hei{学~院~名~称:} \xiaosi\song\underline{\makebox[\@title@width][c]{\@affil}} \\
					\xiaosi\hei{指~导~老~师:} \xiaosi\song\underline{\makebox[\@title@width][c]{\@teacher}}\\
					\xiaosi\hei{学~院~院~长:} \xiaosi\song\underline{\makebox[\@title@width][c]{\@dean}}\\
					\vspace*{0.5cm}
					\xiaosi\song{\makebox[\@title@width][l]{\qquad \qquad \qquad 2019 年 5 月}} \\
					%\end{tabular}
				\end{spacing}
			}
		\end{center}
		
		\clearpage
		\thispagestyle{empty} %去掉页眉页脚
	\end{titlepage}
	\restoregeometry
%  另起一页: 独创性声明和学位论文版权使用授权书
% 如果需要上传稿包含版权页,取消这部分内容注释
%\addcontentsline{toc}{chapter}{学位论文原创性声明和学位论文版权使用授权书}
%\setcounter{page}{1}
%\includepdf{Copyright.pdf}
% 如果需要打印稿,即需要打印后手写,取消该部分内容注释
\pagestyle{fancy}
\fancyhf{}
\fancyfoot[C]{\song\xiaowu ~\thepage~}
\renewcommand{\headrulewidth}{0pt}

\addcontentsline{toc}{chapter}{学位论文原创性声明和学位论文版权使用授权书}{
\setcounter{page}{1}
\qquad\\
\begin{center}\hei\xiaoer{\@untitle}\end{center}\par
\begin{center}\hei\xiaoer{\@declaretitle}\end{center}\par
\song\defaultfont{\@declarecontent}\par
\vspace*{1cm}
{\song\xiaosi
\@authorsigncap \makebox[2.5cm][s]{}
\@signdatecap \makebox[2cm][s]{} 年 \makebox[1cm][s]{} 月 \makebox[1cm][s]{} 日
}
\vspace{0.6\baselineskip}
\begin{center}\hei\xiaoer{\@authorizationtitle}\end{center}\par
{
\vspace{1.2\baselineskip}
\song\defaultfont{\@authorizationcontent}\\
\begin{tabular}{ll}
\song\defaultfont\@authorizationadd\par&\\
&1、保\quad 密\song\xiaoer{$\Box$}\song\xiaosi ,在\underline{\qquad}年解密后适用于本授权书\\
&2、不保密\song\xiaoer{$\Box$}。\\
&(请在以上相应方框内打"$\surd$") \\
\end{tabular}
}
\vspace{2\baselineskip}

{
\song\xiaosi
\@authorsigncap \makebox[3.5cm][s]{}  \@signdatecap \makebox[1.5cm][s]{} 年 \makebox[1cm][s]{} 月 \makebox[1cm][s]{} 日 \\
\indent
\@supervisorsigncap \makebox[3.5cm][s]{}  \@signdatecap \makebox[1.5cm][s]{} 年 \makebox[1cm][s]{} 月 \makebox[1cm][s]{} 日
}
}


%%%%%%%%%%%%%%%%%%%   Abstract and Keywords  %%%%%%%%%%%%%%%%%%%%%%%
\clearpage

\pagestyle{fancy}
  \fancyhf{}
\fancyhead[C]{\song\xiaowu \@heading}
\fancyfoot[C]{\song\xiaowu ~\thepage~}
\makeatletter %双线页眉
\def\headrule{{\if@fancyplain\let\headrulewidth\plainheadrulewidth\fi%
\hrule\@height 1.0pt \@width\headwidth\vskip1pt %上面线为1pt粗
\hrule\@height 0.5pt\@width\headwidth  %下面0.5pt粗
\vskip-2\headrulewidth\vskip-1pt}      %两条线的距离1pt
\vspace{7mm}
}     %双线与下面正文之间的垂直间距

\fancypagestyle{plain}{% 设置开章页页眉页脚风格
    \fancyhf{}%
\fancyhead[C]{\song\xiaowu \@heading}
\fancyfoot[C]{\song\xiaowu ~\thepage~}
}
\addcontentsline{toc}{chapter}{摘\quad 要}
\chapter*{\centering\xiaoer\ 摘\qquad 要}
\song\defaultfont
\@cabstract
\vspace{\baselineskip}

%\hangafter=1\hangindent=52.3pt\noindent   %如果取消该行注释,关键词换行时将会自动缩进
\noindent
{\hei\xiaosi 关键词: \@ckeywords}

%%%%%%%%%%%%%%%%%%%   English Abstract  %%%%%%%%%%%%%%%%%%%%%%%%%%%%%%
\clearpage

\addcontentsline{toc}{chapter}{Abstract}
\chapter*{\centering\xiaoer \bf{Abstract}}
%\vspace{\baselineskip}
\@eabstract
\vspace{\baselineskip}

%\hangafter=1\hangindent=60pt\noindent  %如果取消该行注释,KEY WORDS换行时将会自动缩进
\noindent
{\xiaosi\textbf{Key Words:} \@ekeywords}
}
\clearpage
\makeatother