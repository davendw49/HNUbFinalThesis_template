% !Mode:: "TeX:UTF-8"

\addcontentsline{toc}{chapter}{附录A~~攻读博士学位期间发表的学术论文}%添加到目录中
\chapter*{附录A~~~~攻读博士学位期间发表的学术论文}
\setlength{\parindent}{0em}
\begin{publist}
	\item \textbf{Jianmin Li},Zhaosheng Teng,Qiu Tang,Junhao Song and Xiaoju Li.Dielectric Loss Factor Measurement in Power Systems Based on Sampling Sequence Reconstruction Approach,IEEE Transactions on Power Electronics,2017,32(6):4766-4775. (SCI~收录, 1~区)
	\item \textbf{Jianmin Li},Zhaosheng Teng,Qiu Tang and Junhao Song.Detection and Classification of Power Quality Disturbances Using Double Resolution S-Transform and DAG-SVMs,IEEE Transactions on Instrumentation and Measurement,2016, 65(10):2302-2312.(SCI~收录, 3~区)
	\item \textbf{Jianmin Li},Zhaosheng Teng,Yong Wang,Fu Zhang and Xiaoju Li.A Digital Calibration Approach for Reducing Phase-shift of Electronic Power Meter Measurement,IEEE Transactions on Instrumentation and Measurement.2018, 67(7):1638-1645.(SCI~收录, 3~区)
	\item \textbf{李建闽},滕召胜,吴言,王永.基于移频滤波的频率测量方法.中国电机工程学报,2018.38(3):762-769.(EI~收录,校定重点)
	\item Fu Zhang,Zhaosheng Teng,Yuxiang Yang,\textbf{Jianmin Li},Haowen Zhong and Jiangyan Sang.Near-binary multisine design with arbitrary sparse spectrum for fast bioimpedance spectroscopy measurement.IET Science, Measurement \& Technology.(已录用,SCI~收录,4~区)
	\item Yong Wang,Zhaosheng Teng,He Wen,\textbf{Jianmin Li} and R Martinek.A State Evaluation Adaptive Differential Evolution Algorithm for FIR Filter Design.Advances in Electrical and Electronic Engineering,2018,15(5):770-779.
	\item Fu Zhang,Zhaosheng Teng,Haowen Zhong,Yuxiang Yang,\textbf{Jianmin Li} and Jiangyan Sang. Wideband mirrored current source design based on differential difference amplifier for electrical bioimpedance spectroscopy.Biomedical Physics \& Engineering Express,2018,4(2):25-32.
	\item 邱伟,唐求,唐璐,\textbf{李建闽},何俊杰.基于准同步序列重构的非稳态电力谐波分析.中国电机工程学报,2018,38(2):456-464+676.(EI~收录,校定重点)
	\item 凌菁,滕召胜,林海军,\textbf{李建闽}.烘干失重法水分快速检测的预估融合方法.仪器仪表学报,2018(2):47-55.(EI~收录,校定重点)
	\item 王永,滕召胜,\textbf{李建闽},唐求,成达.基于采样序列重构的高精度介损角测量方法.电工技术学报.(已录用,EI~收录,校定重点)
	\item 滕召胜,王永,\textbf{李建闽},姚文轩,唐求.一种新的谐波时频分解方法——K-S分解.中国科学(技术科学).(已录用,EI~收录)
	\item 李宁,\textbf{李建闽},张建文,宋俊皓,段辉江,荣宏.单相双向计量多功能智能电能表设计.自动化仪表,2017,38(3):70-74.(北大核心)
	\item 吴言,\textbf{李建闽}.基于改进~S~变换的电压骤降的自适应检测方法.电测与仪表.(已录用,北大核心)
	\item 唐夕晴,\textbf{李建闽},佘晓烁.RS-485~总线接口性能测试仪设计与开发.电测与仪表.(已录用,北大核心)
\end{publist}
\addcontentsline{toc}{chapter}{附录B~~攻读博士学位期间参与的科研项目}%添加到目录中
\chapter*{附录B~~~~攻读博士学位期间参与的科研项目}
\setlength{\parindent}{0em}
\begin{publist}
	\item	复杂电网环境下的闪变包络提取与参数在线检测方法研究.国家自然科学基金面上项目(51777061)
	\item	快速~K-S~变换理论与谐波时频参数自适应分析关键技术研究.国家自然科学基金面上项目(51377049)
	\item	改进~S~变换自适应算法与电能质量检测及扰动信号特征提取方法研究.国家自然科学基金面上项目(51277058)
	\item	智能家居电能互动计量与家电远程控制系统.国网新疆电力公司委托项目
\end{publist}
\addcontentsline{toc}{chapter}{附录C~~攻读博士学位期间获得的奖励}%添加到目录中
\chapter*{附录C~~~~攻读博士学位期间获得的奖励}
\setlength{\parindent}{0em}
\begin{publist}
	\item	研究生国家奖学金.国家级(2016.12)
	\item	湖南省普通高校优秀毕业生.省级(2018.05)
	\item	湖南大学优秀研究生.校级(2016.12)
	\item	优秀博士研究生学校奖学金.校级(2017.12)
	\item	湖南大学优秀毕业生.校级(2018.05)
\end{publist}
\vfill
\hangafter=1\hangindent=2em\noindent

\setlength{\parindent}{2em}
